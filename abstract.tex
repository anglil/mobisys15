The new generation of LED-based illuminating infrastructures has enabled a ``dual-paradigm" where LEDs are used for both illumination and communication purposes. The ubiquity of lighting makes visible light communication (VLC) well suited for communication with mobile devices and sensor nodes in indoor environment.
Existing research on VLC has primarily been focused on advancing the performance of one-way communication. %We argue that it is essential to have bi-directional communication capability and simple combination of two one-way VLC is ill-suited for the communication between the lighting infrastructure and a mobile device.
In this paper, we present \retro, a low-power duplex VLC system that enables a mobile device to perform bi-directional communication with the illuminating LEDs over the same light carrier. The design features a retro-reflector fabric that backscatters light, an LCD shutter that modulates information bits on the backscattered light carrier, and several low-power optimization techniques. We have prototyped the \fyi{reader} system and made a few battery-free tag devices. \fyi{Experimental results show that the tag can achieve a $10kbps$ downlink speed and $0.5kbps$ uplink speed over a distance of $2.4m$. We also outline several potential applications of the proposed \retro\ system. } 

%The visible light has been used as a wireless carrier for data communication. Existing designs of Visible Light Communication (VLC) systems, however, consume significant power and only achieve one-directional communications where the mobile device is unable to transmit data to the light bulb on the same band, and hence cannot be applied to low complexity, power-constrained mobile devices or in an interactive manner without occupying extra bandwidth, extremely limiting the use of VLC in mobile and networked settings. This paper makes two main contributions: (1) we introduce the first duplex VLC system design that operates on credit card-sized battery-free devices, and (2) we introduce a communication primitive applicable to secure Radio-Frequency IDentification (RFID) systems acting against side sniffers. 
 %, and (3) we introduce a novel algorithm to detect weak and distorted signals out of strong interferences on the same spectrum that makes the system scalable
% We build a hardware prototype of the above design that can be powered solely using harvested energy from the Light-Emitting Diode (LED). The results show that our design provides benefits for VLC systems and RFID systems: it enables battery-free devices to communicate with LEDs at data rates of $10kbps$ on the downlink and $1kbps$ on the uplink and over a maximum distance of $2.2m$; it limits the uplink signal exposure area within a spindle-shaped area whose radius is less than $0.2m$. We believe that this paper represents a substantial leap in the capability and scalability of VLC systems towards previously infeasible battery-free, duplex, always and anywhere-available and secure ubiquitous communication applications.