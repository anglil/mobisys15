\section{Related Work}


Our work is related to prior work in VLC systems and backscatter communication systems:

\vskip 0.05in\noindent{\bf (a) VLC Systems:} 
%Because LEDs are ubiquitously deployed, there have been efforts to turn them into communication-capable devices. 
Recently, there have been many efforts exploring communication mediums wherein visible lights carry information. 
These \fyi{work}, however, either deal with only one-way communication without an uplink~\cite{flawedsys1,flawedsys2,flawedsys3,flawedsys4}, or go in a two-way fashion with both sides supplied by battery~\cite{led2led1,led2led2,led2led3}, which limit real-world practicality. Specifically, LED-to-phone systems~\cite{location1,location2,location3} only support downlink transmissions, targeted at phone localization. LED-to-LED systems~\cite{led2led4,led2led5} consider visible light networks, where each end is not meant to be mobile, and is not battery-free. 
%LED-Bluetooth systems~\cite{ble0} provide the uplink capability using Bluetooth, but the system occupies an additional band to provide such uplink capability, and is not battery free. 
By contrast, our work augments the existing systems with an additional uplink channel from the mobile device to the LED on the same band as the downlink, with an emphasis on the low power design and system robustness. %instead of boosting data rate.
%while eliminating the need for battery supply for the mobile device.

%\p{RFID, then ambientscatter. Amb scatter is for communication between devices, not to communication based to the energy source. Be our future work. }
\vskip 0.05in\noindent{\bf (b) Backscatter Systems:} 
Backscattering is a way to provide transmission capability for extremely low-power devices, substituting the need for devices actively generating signals. The technique has been primarily used by RFID tags~\cite{rfid1,rfid2}. Recently, Wi-Fi ~\cite{abc3} and TV-based ~\cite{abc1,abc2} systems started employing and advancing this technique. 


Our \retro\ system also achieves low-energy design using backscattering and further shares design principles with \cite{abc1,abc2, abc3}, that is, using analog components on the energy-constrained end. The major differences lie in the fact that we are dealing with visible light using a retro-reflector, whereas the ambient backscatter systems are backscattering radio waves. On the tag side, we use a light sensor to receive and a retro-reflector to send (by reflection) information, which is also different from the shared antenna and RF front-end in other backscattering systems. In comparison, we can easily achieve full-duplex while other systems are essentially half-duplex and require intensive tricks and significant overhead to achieve full-duplex \cite{fullduplex1,fullduplex2,fullduplex3}. 

%Due to the ubiquity of light infrastructure, our system can be widely applied, especially in indoor environments, whereas TV signal backscattering~\cite{abc1,abc2} systems are limited to areas close to TV tower, and Wi-Fi backscattering~\cite{abc3} further limits the communication range to be very close to the AP and can only work intermittently due to Wi-Fi signal's bursty nature. The TV-based backscattering systems aims at enabling communication among devices, instead of communication back to the infrastructure. 

%RFID systems~\cite{rfid1,rfid2} typically include a passive tag, too. However, first, RFID systems need a second-order modulation on the tag to eliminate base band noise, while \vitag's uplink does not have any second-order modulation, relying on the \reader\ who conducts demodulation and decoding algorithms to extract useful information. Second, RFID readers are not ubiquitous as LEDs, and readers are not readily networked like LEDs~\cite{flawedsys1}. Finally, in terms of capability, in comparison with NFC RFID systems~\cite{iso1}, visible lights can transmit farther in the line-of-sight scenario; as for vicinity RFID systems~\cite{iso2}, visible lights transmit more stably, in the sense that a transmission is not as easily distorted by surrounding object movements.

%TV signal backscattering~\cite{abc1,abc2} works at a band around $539MHz$, leveraging the existing TV signals to power up small devices to do communications. However, the locations at which these systems can work are extremely limited. In typically indoor environments, TV signals cannot be detected everywhere~\cite{abc2}. Plus, there isn't any uplink from these devices to TV towers.

%Wi-Fi backscattering~\cite{abc3} is an alternative that provides both downlink and uplink transmissions for battery-free devices around Wi-Fi routers. However, due to Wi-Fi signals' bursty nature, the communication is not really available all the time; when there is no ongoing Wi-Fi packets, neither can enough radio energy be harvested nor are there signals to be modulated by devices.

%Finally, in general, these backscattering systems tend to expose their transmissions to a wide area, because of the scattering nature. This potentially gives side readers a chance to overhear the information being transmitted~\cite{abc1,abc2,abc3}. By contrast, \vitag relies on visible light communication, which implies eavesdroppers are easily discernible. The use of retro-reflectors further retains the uplink transmission along the tag-reader path. As a result, our system comes with a good security property inherently while systems on other bands have to enhance their security with extra efforts~\cite{eavesdrop1,eavesdrop2}.

In addition, because of the backscattering nature, these wireless systems tend to expose their transmissions to a wide surrounding area, leaving a good chance for side readers to overhear the information being transmitted~\cite{abc1,abc2,abc3}. By contrast, \vitag relies on visible light communication, which implies that eavesdroppers are easily discernible. The use of retro-reflectors further constraints the uplink transmission to stick along the tag-reader path. As a result, our system \vitag comes with a good security property inherently, while other systems have to enhance their security with extra efforts~\cite{eavesdrop1,eavesdrop2}.


% \vskip 0.05in\noindent{\bf (c) Duplex Systems:} Previously, it is either the case that systems built on visible light bands are non-duplex~\cite{flawedsys1}, or that duplex systems are not both operating at visible light frequencies~\cite{ble1,ble2,ble3,ble4}. Moreover, radio systems that achieve the full-duplex feature require intensive interference cancellation and noise filtering tricks that consume a huge amount of energy on the backscatter side, of which FFT and dynamic adaption are the most energy demanding components~\cite{fullduplex1,fullduplex2,fullduplex3}. In the \vitag\ design, we avoid heavy energy consumption by focusing the uplink energy better at the LED receiver. In lieu of the use of complex digital signal processing components by current designs on the back-scatterer end~\cite{fullduplex4,fullduplex5}, the careful design of an error-restricting time recovery algorithm in our design helps hugely remove the interference caused by reflection of the signals transmitted by \reader\ itself, adapting the systems to almost all kinds of lightening conditions. We also design an energy collection module that retrieves the residual energy in the course of LCD discharge, such that the tag size can be minimized while retaining the battery-free feature.

% At TV frequencies, a recent work on full-duplex backscatter system~\cite{abc5} uses different frequencies for the uplink and downlink, respectively, on the signal envelope. However, it lacks scalability and is not resource-efficient. \vitag\, as mentioned earlier, has focused uplink transmissions and does not differentiate the carrier frequency between the uplink and downlink, easily scalable to a network with multiple LEDs and mobile devices. 