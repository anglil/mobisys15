\section{Media Access Control}
\label{sec:mac}

The discussion so far focuses on the communication aspects of a single \vitag-\reader\ pair. However, when many of these devices are in range of each other, we need mechanisms to arbitrate the channel between them. Unlike traditional RFID, the communication uplink from \vitag\ to \reader\ is highly directional because of the retro-reflectors. In addition, as a system with multiple access points that connect to the Internet, which is also different from RFID systems, \vitag\ needs mechanisms to provide roaming support. In this section, we explore the Media Access Control (MAC) design for \vitag\ and \reader\ with a break-down into four scenarios, namely, one \reader\ to multiple \vitag\/s, multiple \vitag\/s to one \reader, multiple \reader\/s to one \vitag, and one \vitag to multiple \reader\/s.

\subsection{One \reader\ to Multiple \vitag\/s}
\label{subsec:onereaderto}
One of the problems is how a \reader\ identifies a \vitag\ with a specific serial number from a number of \vitag\/s in range. This is necessary because if multiple \vitag\/s respond simultaneously to a query from \reader, they will jam each other. In \vitag, we set all \vitag\/s in a passive state, waiting for polling requests sent by \reader. When the serial number of a tag is called, the tag with this serial number responds within an assigned time slot. The rest of the \vitag\/s will ignore the payload that follows the serial number in the query as they notice that the serial number do not align with their own. For the requested \vitag\ to respond, it only needs to modulate the LCD and \textit{directionally} sends information back to the \reader\ that initiated the conversation. Other \vitag\/s and \reader\/s nearby will not hear anything from the requested \vitag.

\subsection{Multiple \vitag\/s to One \reader}
\label{subsec:multitagto}
When multiple \vitag\/s want to talk to one \reader\ simultaneously, every \vitag\ has to wait for it's own time slot scheduled by the \reader\ to transmit.

\subsection{Multiple \reader\/s to One \vitag}
\label{subsec:multireaderto}
One other problem is when multiple \reader\/s want to talk to one \vitag, how to arbitrate the media. To solve this problem, \reader\/s run ALOHA with carrier sensing. Specifically, if \reader\ has data to send, send the data. If, while \reader\ is transmitting data, it receives any data from another \reader, there has been a message collision, in which case all involved \reader\/s back off for an arbitrary period of time before retrying. Unlike \vitag, \reader\/s do not have a tight energy constraint, and so carrying out carrier sensing on them is possible.

\subsection{One \vitag\ to Multiple \reader\/s}
\label{onetagto}
The reverse problem is when \vitag\ wants to talk to one \reader, how it will prevent other \reader\/s in range from being interrupted. In principle, \vitag\ is supposed to respond to the polling request sent by the \reader\ that has the strongest illuminance on \vitag, so as to get the best performance. However, detecting light strength is too energy consuming for \vitag\/s. On the other hand, \reader\ does not have a tight energy budget and can assess the strength as well, of the signal backscattered by \vitag\/s, which is negatively correlated with the distance from a \vitag\ to a \reader. To provide \vitag\/s in the network with the best connection, \reader\/s estimate the accessibility of every \vitag\ in range using the feedback signal in each \vitag's time slot. Specifically, the network of \reader\/s works out a mapping between best service-providing \reader\/s and every \vitag\ in range\footnote{One could exploit the link-state routing protocol to achieve  consensus on such a mapping across \reader\/s}, and keeps this information in each \reader's "\vitag\ table". Now, as every \reader\ knows which \vitag\ to serve to get the best performance, it will send polling requests only to \vitag\/s that are in the instantaneous "\vitag\ table".

\subsection{Putting Things Together}
The discussion so far brings together the physical layer and the MAC layer of the \vitag\ system. The physical layer protocol deals with point-to-point communications on the downlink (from one \reader\ to one \vitag) and the uplink (from one \vitag\ to one \reader) as well as \vitag\ duty-cycling, \vitag\ waking-up, and error-correction. The MAC layer protocol addresses the multi-\reader\ to multi-\vitag\ problem in a way different from existing RFID or WLAN because of the different constraints. With these protocols, the networked system of \vitag\/s can provide services like Internet connection to batter-free tags in the home-area sensor network scenario, and identification service in traditional RFID scenarios, with better security guarantees. We will showcase the prototype we build for the RFID scenario, and evaluate its performances and security preservation strength. 